\documentclass[11pt]{article}

\usepackage[utf8]{inputenc}
\usepackage[T1]{fontenc}
\usepackage{microtype}
\usepackage{amsmath}

\newcommand{\bigo}{O}
\newcommand{\smallo}{o}

\title{Succinct data structures}
\author{Jani Rahkola}

\begin{document}
\maketitle

\section{Introduction}

Succinct data structures store data using near optimal amount of bits
while still enabling efficient operations. The difference to
compressed data structures is that succinct data structures allow
operations without a preceding decompression. Also, the space effiency
of the succinct data structures does not depend on the values of the
input data, as might be the case with compressed data structures.

Succinct data structures are divided into three groups based on how
close they get to the optimal amount of space. Say $T$ is the optimal
amount of bits needed to store the actual data. Now an structure
taking $T + \bigo(1)$ bits is called \emph{implicit}. Classical
examples of implicit data structures are the sorted array and the
array backed binary heap.

As a double use of the term, a data structure taking $T + \smallo(T)$
bits is called \emph{succinct}. Classical example of a succinct data
structure is the bit vector with constant time rank and select.
\end{document}
